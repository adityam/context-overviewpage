\setupinteraction[state=start]

% Minimalistic subsection layout
\setuppapersize[S4]
\setupwhitespace[medium]
\setuphead[section, subsection][color=blue, page=yes]

\setupbodyfont[dejavu, sans, 12pt]

\usemodule[overviewpage][level=section]

% Filler text
\startbuffer[subsection]
  \startsubsection[title={Some random bullet points}]
    \startitemize
      \item First point
      \item Second point
      \item Third point
      \item And so on \unknown
    \stopitemize
  \stopsubsection
\stopbuffer

\starttext

\startsection[title={First topic in the presentation}]

  \startplacefigure[location={here,nonumber}, title={A cute cat}]
    \externalfigure[http://placekitten.com/g/800/300][method=jpg, width=0.7\textwidth]
  \stopplacefigure

  According to the above layout, a topic consists of multiple subsections. Think of
  it as a \type{section} in \type{beamer}. 

  \dorecurse{4}{\getbuffer[subsection]}

\stopsection

\startsection[title={Second topic in the presentation}]
  \startplacefigure[location={here,nonumber}, title={A cute cat}]
    \externalfigure[http://placekitten.com/g/800/350][method=jpg, width=0.7\textwidth]
  \stopplacefigure

  Another topic and some random stuff to explain about this topic. 

  \dorecurse{4}{\getbuffer[subsection]}

\stopsection

\startsection[title={Third topic in the presentation}]
  \startplacefigure[location={here,nonumber}, title={A cute cat}]
    \externalfigure[http://placekitten.com/g/800/380][method=jpg, width=0.7\textwidth]
  \stopplacefigure

  Another topic and some random stuff to explain about this topic. 

  \dorecurse{4}{\getbuffer[subsection]}

\stopsection

\startsection[title={Fourth topic in the presentation}]
  \startplacefigure[location={here,nonumber}, title={A cute cat}]
    \externalfigure[http://placekitten.com/g/800/325][method=jpg, width=0.7\textwidth]
  \stopplacefigure

  Another topic and some random stuff to explain about this topic. 

  \dorecurse{4}{\getbuffer[subsection]}

\stopsection

\placeoverviewpage

\stoptext

